\documentclass{ltxguidex}
% for demonstrating translations
\usepackage[german, american]{babel}
%\usepackage{translations}

\usepackage{fontspec}
\errorcontextlines=10
\usepackage{changelog}
\usepackage{changepage}
\newcommand{\cl}{\ctan{changelog}}
\newcommand{\keepachangelog}{\https{keepachangelog.com}}

\makeatletter
\renewcommand{\SX@explpreset}{
	language=[LaTeX]TeX,
	numbers=none,
}
\makeatother

\setmainfont{Charter}
\usepackage{FiraSans}
\usepackage{FiraMono}
\lstset{basewidth=0.6em}


\NewDocumentEnvironment{ctandescription}{}
	{\NewDocumentCommand{\pkg}{m}{\item[\ctan{##1}]}
	\begin{description}}
	{\end{description}}

\author{Rebecca Turner\thanks{\email{rbt@sent.as}}}
\title{The \cl\ Package}
\date{${VERSION}$}
\begin{document}
\maketitle

\begin{abstract}
	Changelogs are important. Unfortunately, there are few facilities
	for typesetting changelogs in \LaTeX. \cl\ defines a |changelog|
	environment to make changelogs simple and intuitive.
	
	For rationale, read \keepachangelog.
\end{abstract}

\begin{note}
	Browse the sources, contribute, or complain at \\
	\https{github.com/9999years/latex-changelog}
\end{note}

\tableofcontents
\vfill
\pagebreak

\section{I don't want to read this documentation}

Perfectly understandable. \cl\ comes in a ``simple'' flavor:

\begin{LTXexample}
\begin{changelog}[author=Rebecca Turner, simple, sectioncmd=\section*]
\begin{version}[v=1.0.0, date=2018-12-28]
  \item Cool features
  \item Bug fixes
\end{version}
\shortversion{v=0.1.0, date=2018-10-25, changes=Initial beta}
\end{changelog}
\end{LTXexample}

And a more organized variant based on \keepachangelog:

\begin{LTXexample}
\begin{changelog}[author=Rebecca Turner, sectioncmd=\section*]
\begin{version}[v=1.0.0, date=2018-12-28]
  \added
    \item Cool features
  \fixed
    \item Bug fixes
\end{version}
\shortversion{v=0.1.0, date=2018-10-25, changes=Initial beta}
\end{changelog}
\end{LTXexample}

\section{Intro}

\subsection{Why?}

Read \href{https://olivierlacan.com/}{Olivier Lacan}'s lovely site
\keepachangelog. To excerpt:

\begin{adjustwidth}{1in}{1in}
	\subsubsection{What is a changelog?}
	A changelog is a file which contains a curated, chronologically
	ordered list of notable changes for each version of a project.

	\subsubsection{Why keep a changelog?}
	To make it easier for users and contributors to see precisely what
	notable changes have been made between each release (or version) of
	the project.

	\subsubsection{Who needs a changelog?}
	People do. Whether consumers or developers, the end users of
	software are human beings who care about what's in the software.
	When the software changes, people want to know why and how.
\end{adjustwidth}

\subsection{The competition}

It's always good to know the competition. Unfortunately, there isn't much
here. Know of another package with similar functionality? Drop me a line or
open a pull request!

\begin{ctandescription}
	\pkg{vhistory} provides a decent-looking changelog. However, it's
		designed for short changes and provides a less-than-elegant
		interface. Additionally, it's based on the \ctan{ltxtable}
		package, meaning it makes restrictions on the contents of
		version information and writes the table to a file.

	\pkg{holtxdoc} has a decent changelog feature (via
		Oberdiek's |History| and |Version| environments),
		but \ctan{holtxdoc} ``contains some private macros and setup
		for my needs. Thus do not use it.'' In addition, Oberdiek's
		changelogs don't support multiple authors.

	\pkg{gitlog} is an interesting idea, but \ctan{gitlog} ``is a
		proof-of-concept release to allow users an early
		evaluation\dots''

		Also,
		\href{https://keepachangelog.com/en/1.0.0/}{friends don't
		let friends dump git logs into changelogs}.
\end{ctandescription}

\section{The \texttt{changelog} environment}

\begin{desc}
|\begin{changelog}[<options>]...\end{changelog}|
\end{desc}
Wraps |description| while providing the |version| environment and the
\cs{shortversion} command. In addition to the options shown below,
\meta{options} may contain any of the options for |version| as a form of
``partial application''; this may be useful if, for example, most of your
versions have the same author.

\begin{keys}
	\key{section}[\bool][true]
		Insert a \cs{section} before the changelog?
	\key{sectioncmd}[\m{command}][\cs{section}]
		Which sectioning command to use?
	\key{title}[\m{txt}][Changelog]
		What to title the changelog section?
	\key{label}[\m{label}][sec:changelog]
		What to \cs{label} the section?
\end{keys}

\begin{desc}
|\begin{version}[<options>]...\end{version}|
\end{desc}
Gives a single version; wraps |itemize|.

If the \option{date} option is absent, the date isn't printed.

If the \option{v}/\option{version} option is absent, the date is used in its
place.

If both \option{version} and \option{date} are absent, the version is shown
as \texttt{Unreleased} and \cs{today} is used for the date. \cs{today} isn't
ideal (which is to say, not
\href{https://en.wikipedia.org/wiki/ISO_8601}{\textsc{iso} 8601} compliant)
but it's well-known and easy to redefine.\footnote{Try the \ctan{datetime2}
package or
\href{https://tex.stackexchange.com/questions/152392/date-format-yyyy-mm-dd}{any
of the other solutions here.}}

\m{options} may include:

\begin{keys}
	\key{version}[\m{version string}]
		The version string for this version
	\key{v}[\m{version string}]
		An alias for \option{version}
	\key{author}[\m{author(s)}]
		The author(s) of this version
	\key{date}[\m{date string}]
		The date of this version's release
	\key{yanked}[\bool][false]
		Indicates that the release was revoked due to a ``serious bug or
		security issue''; prints a visible notice next to the version number
	\key{simple}[\bool][false]
		Indicates this version isn't split up into \cs{added}, \cs{changed},
		etc.\ categories; if this option is given, a |version| environment acts
		like a plain |itemize|
	\key{short}[\bool][false]
		Renders a short version, like |\shortversion|'s \option{changes} option;
		useful if you need to put verbatim material in the changes which would
		otherwise cause an error due to being used in an argument.
\end{keys}

\begin{desc}
|\shortversion{<options>}|
\end{desc}
A short, one-line version. In addition to |version|'s options, the following
options are available for \cs{shortversion}:

\begin{keys}
	\key{changes}[\m{change text}] The changes to display for this version
\end{keys}

\subsection{Helper commands}

The \cl\ package defines several ``helper commands.'' These commands
introduce a \textit{set} of changes within the version. There must be at
least one \cs{item} between one of these commands and the end of the
|version| environment.

Note that these commands are \emph{only} available when not using
the \option{simple} option!

\begin{desc}|\added|\end{desc}Introduces a list of \cs{item}s that represent added
	features
\begin{desc}|\changed|\end{desc}Introduces a list of \cs{item}s that represent
	changed features
\begin{desc}|\deprecated|\end{desc}Introduces a list of \cs{item}s that represent
	deprecated features
\begin{desc}|\removed|\end{desc}Introduces a list of \cs{item}s that represent
	features which have been removed
\begin{desc}|\fixed|\end{desc}Introduces a list of \cs{item}s that represent bug
	fixes
\begin{desc}|\security|\end{desc}Introduces a list of \cs{item}s that represent
	security-fixes and closed security holes

\subsection{Customization}

The |changelog| environment wraps |changelogdescription| (which is
defined by default to be just the |description| environment), and the
|version| environment wraps |changelogitemize|. One could customize
these in depth with \ctan{enumitem}, as in:

\begin{latexcode}
\usepackage{enumitem}
\renewenvironment{changelogitemize}
	{\begin{itemize}[label=---]}
	{\end{itemize}}
\end{latexcode}

\begin{desc}
|\changelogyanked|
\end{desc}
Prints the ``revoked release'' notice:
\changelogyanked. See:
\href{https://keepachangelog.com/en/1.0.0/#yanked}{yanked releases on
\texttt{keepachangelog.com}}.

\section{Translations}

Want to use \cl\ in a non-English document? Great! \cl\ comes with
translations for English, German, Spanish, French, and Japanese, as well as
support for adding new translations (see section~\ref{s:contrib} below for
translation credits). To use built-in translations, simply load \ctan{babel}
with the desired language:
\begin{otherlanguage}{german}
\begin{LTXexample}
% In the preamble:
% \usepackage[german]{babel}

\begin{changelog}[section=false, author=Holger Schieferdecker]
  \begin{version}[v=1.0.0]
    \added
      \item Features!
  \end{version}
\end{changelog}
\end{LTXexample}
\end{otherlanguage}

\subsection{Creating new translations}

\cl\ uses the \ctan{translations} package to provide translations, and a
complete translation looks like this:
\begin{latexcode}
\documentclass{article}
\usepackage[english]{babel}
\usepackage{translations}
\DeclareTranslation{English}{changelog-Added}{Added}
\DeclareTranslation{English}{changelog-Changed}{Changed}
\DeclareTranslation{English}{changelog-Deprecated}{Deprecated}
\DeclareTranslation{English}{changelog-Removed}{Removed}
\DeclareTranslation{English}{changelog-Fixed}{Fixed}
\DeclareTranslation{English}{changelog-Security}{Security}
\DeclareTranslation{English}{changelog-Miscellaneous}{Miscellaneous}
\DeclareTranslation{English}{changelog-Unreleased}{Unreleased}
\DeclareTranslation{English}{changelog-Yanked}{YANKED}
\begin{document}
% ...
\end{document}
\end{latexcode}
Refer to the
\href{http://mirrors.ctan.org/macros/latex/contrib/translations/translations_en.pdf}{\texttt{translations}
package documentation} and the
\href{http://mirrors.ctan.org/macros/latex/required/babel/base/babel.pdf}{\texttt{babel}
package documentation} for more information.

If you create a translation, please \href{mailto:rbt@sent.as}{send me
an email} and I'll incorporate the translation into \cl's next release!

\section{Contributors}%
\label{s:contrib}

\begin{description}
	\item[Rebecca Turner] Original implementation.
	\item[Holger Schieferdecker]
	Internationalization support and German translations.
	\item[David Arnold <\email{dar@xoe.solutions}>]
	Support for translating the changelog's section title and Spanish
	translations.
	\item[Damien Calesse <\https{github.com/kranack}>] French translations.
	\item[cmplstofB <\https{github.com/cmplstofB}>] Japanese translations.
\end{description}

\section{Changelog}

This is this package's actual changelog --- not an example!

\begin{changelog}[author=Rebecca Turner <\email{rbt@sent.as}>, section=false]
\begin{version}[v=2.4.0, date=2020-09-02,
	author={cmplstofB <\https{github.com/cmplstofB}>}]
\added
	\item Japanese translations.
\end{version}

\begin{version}[v=2.3.0, date=2020-08-26,
	author={Damien Calesse <\https{github.com/kranack}>}]
\added
	\item French translations.
\end{version}

\begin{version}[v=2.2.0, date=2020-04-23,
	author={David Arnold <\email{dar@xoe.solutions}>}]
\added
	\item Support for translating the changelog's section title with
	|\DeclareTranslation{Language}{changelog}{...}|.
	\item Spanish translations.
\end{version}

\begin{version}[v=2.1.0, date=2019-06-29,
	author={Holger Schieferdecker}]
\added
	\item Internationalization support.
	\item German translations.
\end{version}

\begin{version}[v=2.0.0, date=2019-04-15]
\added
	\item Better error handling; more informative messages for empty
	\texttt{version} or \texttt{changelog} environments.
	\item \option{short} option in versions.
\removed
	\item Removed \texttt{color} package option.
\fixed
	\item Added dependencies for \ctan{xparse} and \ctan{xkeyval}, which had
	erroneously been excluded from the package.
\end{version}

\begin{version}[v=1.0.0, date=2018-12-28]
\added
	\item The ``simple'' option for changelogs which aren't split up
	into sections of added, changed, removed, etc.\ features
\changed
	\item Instead of commands like \cs{added} introducing an item, they
	introduce a list of items; this is what \keepachangelog\
	actually intended; previous implementations were incorrect to
	display an ``Added'' (or whatever) marker next to each item.
\end{version}

\shortversion{v=0.2.1, date=2018-10-26, changes=Documented \option{yanked}
	option}
\shortversion{v=0.2.0, date=2018-10-26, changes=First stable release}
\shortversion{v=0.1.0, date=2018-10-25, changes=Initial beta}
\end{changelog}

\end{document}
