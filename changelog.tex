\documentclass{ltxdoc}
\usepackage{changelog-doc}
\author{Rebecca Turner\thanks{Brandeis University; \email{rebeccaturner@brandeis.edu}}}
\title{The \cl\ Package}
\date{2018-10-26}

\begin{document}
\maketitle

\begin{abstract}
	Changelogs are important. Unfortunately, there are few facilities
	for typesetting changelogs in \LaTeX. \cl\ defines a \env{changelog}
	environment to make changelogs simple and intuitive.
	
	For rationale, read \https{keepachangelog.com}.
\end{abstract}

\noterepo{9999years/latex-changelog}

\tableofcontents
\vfill
\pagebreak

\section{Intro}

\subsection{Why?}

Read \href{https://olivierlacan.com/}{Olivier Lacan}'s lovely site
\https{keepachangelog.com}. To excerpt:

\begin{adjustwidth}{1in}{1in}
	\subsubsection{What is a changelog?}
	A changelog is a file which contains a curated, chronologically
	ordered list of notable changes for each version of a project.

	\subsubsection{Why keep a changelog?}
	To make it easier for users and contributors to see precisely what
	notable changes have been made between each release (or version) of
	the project.

	\subsubsection{Who needs a changelog?}
	People do. Whether consumers or developers, the end users of
	software are human beings who care about what's in the software.
	When the software changes, people want to know why and how.
\end{adjustwidth}

\subsection{The competition}

It's always good to know the competition. Unfortunately, there isn't much
here. Know of another package with similar functionality? Drop me a line or
open a pull request!

\begin{ctandescription}
	\pkg{vhistory} provides a decent-looking changelog. However, it's
		designed for short changes and provides a less-than-elegant
		interface. Additionally, it's based on the \ctan{ltxtable}
		package, meaning it makes restrictions on the contents of
		version information and writes the table to a file.

	\pkg{holtxdoc} has a decent changelog feature (via
		Oberdiek's \env{History} and \env{Version} environments),
		but \ctan{holtxdoc} ``contains some private macros and setup
		for my needs. Thus do not use it.'' In addition, Oberdiek's
		changelogs don't support multiple authors.

	\pkg{gitlog} is an interesting idea, but \ctan{gitlog} ``is a
		proof-of-concept release to allow users an early
		evaluation\dots''

		Also,
		\href{https://keepachangelog.com/en/1.0.0/}{friends don't
		let friends dump git logs into changelogs}.
\end{ctandescription}

\section{The \env{changelog} environment}

\begin{macro}{changelog}\oarg{options}\AfterLastParam Wraps
\env{description} while providing the \env{version} environment and the
\cs{shortversion} command. In addition to the options shown in
table~\ref{opt:changelog}, \meta{options} may contain any of the options for
\env{version} (see figure~\ref{opt:version}) as a form of ``partial
application''; this may be useful if, for example, most of your versions
have the same author.

\begin{table}[h]
	\centering
	\caption{Options for the \env{changelog} environment}
	\label{opt:changelog}
	\begin{adjustwidth}{1in}{1in}
	\begin{Optionlist}
		section & Insert a \cs{section} before the changelog?
			Default: true \\
		sectioncmd & Which sectioning command to use? Default:
			\cs{section} \\
		title & What to title the changelog section? Default:
			\texttt{Changelog} \\
		label & What to \cs{label} the section? Default:
			\texttt{sec:changelog} \\
	\end{Optionlist}
	\end{adjustwidth}
\end{table}
\end{macro}

\begin{macro}{version}\oarg{options}\AfterLastParam
Gives a single version; wraps \env{itemize}.

If the \option{date} option is absent, the date isn't printed.

If the \option{v}/\option{version} option is absent, the date is used in its
place.

If both \option{version} and \option{date} are absent, the version is shown
as \texttt{Unreleased} and \cs{today} is used for the date. \cs{today} isn't
ideal (which is to say, not
\href{https://en.wikipedia.org/wiki/ISO_8601}{\textsc{iso} 8601} compliant)
but it's well-known and easy to redefine.\footnote{Try the \ctan{datetime2}
package or
\href{https://tex.stackexchange.com/questions/152392/date-format-yyyy-mm-dd}{any
of the other solutions here.}}

\begin{table}[h]
	\centering
	\caption{Options for the \env{version} environment}%
	\label{opt:version}
	\begin{adjustwidth}{1in}{1in}
	\begin{Optionlist}
		version & The version string for this version \\
		v & An alias for \option{version} \\
		author & The author(s) of this version \\
		date & The date of this version's release \\
		yanked & Indicates \\
	\end{Optionlist}
	\end{adjustwidth}
\end{table}

\end{macro}

\begin{macro}{\shortversion}\marg{options}\AfterLastParam
A short, one-line version. In addition to the options specified in
table~\ref{opt:version}, the following options are available for
\cs{shortversion}:

	\begin{Optionlist}
		changes & The changes to display for this version \\
	\end{Optionlist}
\end{macro}

\subsection{Helper commands}

The \cl\ package defines several ``helper commands,'' which are colorized if
the \option{color} package option has been given. See
section~\ref{sec:colors} for more information. These commands include a
trailing space; |\added a cool feature| will print with a space between
``Added'' and ''a cool feature''.

\begin{macro}{\added}Prints an \cs{item} beginning with ``Added''\end{macro}
\begin{macro}{\changed}Prints an \cs{item} beginning with ``Changed''\end{macro}
\begin{macro}{\deprecated}Prints an \cs{item} beginning with ``Deprecated''\end{macro}
\begin{macro}{\removed}Prints an \cs{item} beginning with ``Removed''\end{macro}
\begin{macro}{\fixed}Prints an \cs{item} beginning with ``Fixed''\end{macro}
\begin{macro}{\security}Prints an \cs{item} beginning with ``Security''\end{macro}

\subsection{Customization}

The \env{changelog} environment wraps \env{description}, and the
\env{version} environment wraps \env{itemize}. One could customize these in
depth with \ctan{enumitem}.

\subsection{Colors}%
\label{sec:colors}

Colored output is supported by \ctan{xcolor}, which defines several named
colors (as seen in figure~\ref{fig:colors}); these colors may be redefined
as needed. While the \ctan{xcolor} documentation goes into great detail,
you'll likely do fine with just e.g.\ \cs{colorlet}|{ChangelogAdded}{magenta}|.

\begin{figure}[h]
	\centering
	\begin{adjustwidth}{1in}{1in}
	\begin{colorlist}
		\item{ChangelogAdded}
		\item{ChangelogChanged}
		\item{ChangelogDeprecated}
		\item{ChangelogRemoved}
		\item{ChangelogFixed}
		\item{ChangelogSecurity}
	\end{colorlist}
	\end{adjustwidth}
	\caption{Colors provided by the \cl\ package}
	\label{fig:colors}
\end{figure}

\subsubsection{Package options}

\begin{Optionlist}
	color & Makes output more colorful; this is probably not that useful
\end{Optionlist}

\begin{changelog}[author=Rebecca Turner,
	sectioncmd=\subsection,
	title=Example changelog]
\begin{version}
	\added Really cool features
\end{version}

\begin{version}[date=2019-01-23]
	\item A version with only a date
\end{version}

\begin{version}[v=1.1.0]
	\item A version with no date
\end{version}

\begin{version}[v=1.0.1, yanked]
	\item A version with a terrible bug
\end{version}

\begin{version}[v=1.0.0, date=2018-10-26]
	\added a cool feature
	\changed some \textsc{api} detail
	\deprecated something that was a bad idea in the first place
	\removed something that was deprecated 3 versions ago
	\fixed a bug that would delete files instead of saving them
	\security improved with addition of buffer bound checks
	\item A change that doesn't fit into any other category
\end{version}
\shortversion{v=0.1.0, date=2018-10-19,
	changes=Initial beta.}
\end{changelog}

Which is produced with:

\begin{latexcode}
\usepackage[color]{changelog}
%...
\begin{changelog}[author=Rebecca Turner,
	sectioncmd=\subsection,
	title=Example changelog]
\begin{version}
	\added Really cool features
\end{version}

\begin{version}[date=2019-01-23]
	\item A version with only a date
\end{version}

\begin{version}[v=1.1.0]
	\item A version with no date
\end{version}

\begin{version}[v=1.0.1, yanked]
	\item A version with a terrible bug
\end{version}

\begin{version}[v=1.0.0, date=2018-10-26]
	\added a cool feature
	\changed some \textsc{api} detail
	\deprecated something that was a bad idea in the first place
	\removed something that was deprecated 3 versions ago
	\fixed a bug that would delete files instead of saving them
	\security improved with addition of buffer bound checks
	\item A change that doesn't fit into any other category
\end{version}
\shortversion{v=0.1.0, date=2018-10-19,
	changes=Initial beta.}
\end{changelog}
\end{latexcode}

\section{Changelog}

This is this package's actual changelog --- not an example!

\begin{changelog}[author=Rebecca Turner, section=false]
\shortversion{v=0.2.0, date=2018-10-26, changes=First stable release}
\shortversion{v=0.1.0, date=2018-10-25, changes=Initial beta}
\end{changelog}

\PrintIndex
\listoftables

\end{document}
